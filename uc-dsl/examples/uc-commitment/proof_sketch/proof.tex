\def\IsDraft{} % set for draft version

\newif\iffull
\fulltrue
%\fullfalse

\documentclass[11pt,letterpaper]{article}

% == Import Packages ==
\usepackage[draft,notes=true,later=false]{dtrt}
\usepackage[margin=1in]{geometry} % Page dimensions
\usepackage{amssymb,amsfonts,amsmath,amsthm} % Math formulas
\usepackage{xcolor} % Colors
\usepackage{listings} % Listings, for code snippets
\usepackage{hyperref} % Hyperlinks
\usepackage{enumitem}
\usepackage[capitalise]{cleveref} % Clever References, must be loaded AFTER all other packages

% == Formatting ==
\renewcommand{\emph}[1]{\textbf{#1}~}
\newcommand{\doclearpage}{%
  \iffull\clearpage\else\fi
}

% == Inline comments ==
\ifdefined\IsDraft
\newcommand{\authnote}[2]{[{\color{red}\textbf{#1:}}~{\color{blue} #2}]}
\else
\newcommand{\authnote}[2]{}
\fi

\newcommand{\alley}[1]{\authnote{Alley}{#1}}
\newcommand{\megan}[1]{\authnote{Megan}{#1}}
\newcommand{\mayank}[1]{\authnote{Mayank}{#1}}
\newcommand{\ran}[1]{\authnote{Ran}{#1}}

%%%%%%%%%%%%%%%%%%%%%%%%%%%%%%%%%%%%%%%%%%%%%%%%%%%%%%%%%%%%%%%%%%%%%%%%%%%%%%%%
% THEOREMS
\theoremstyle{plain} % italics
\newtheorem{itheorem}{Theorem}%[section]
\newtheorem{idefinition}{Definition}%[section]
\newtheorem{icorollary}{Corollary}%[section]
\newtheorem{ilemma}{Lemma}%[section]
\iffull
\newtheorem{theorem}{Theorem}[section]
\newtheorem{lemma}[theorem]{Lemma}
\newtheorem{proposition}[theorem]{Proposition}
\newtheorem{claim}[theorem]{Claim}
\newtheorem{definition}[theorem]{Definition}
\fi
\newtheorem{corollary}[theorem]{Corollary}
\newtheorem{assumption}{Assumption}
\newtheorem{observation}[theorem]{Observation}

\theoremstyle{definition} % not italics
\newtheorem{construction}[theorem]{Construction}
\iffull
\newtheorem{remark}[theorem]{Remark}
\newtheorem{example}[theorem]{Example}

\theoremstyle{remark} %
\newtheorem{case}{Case}
\fi
\newtheorem{fact}[theorem]{Fact}


%%%%%%%%%%%%%%%%%%%%%%%%%%%%%%%%%%%%%%%%%%%%%%%%%%%%%%%%%%%%%%%%%%%%%%%%%%%%%%%%
% == Code snippets ==
% Listings package docs: http://texdoc.net/texmf-dist/doc/latex/listings/listings.pdf

\lstdefinelanguage{easycrypt}{
	morekeywords={lemma, module, proc, var, return, if, else, require, import, type, op, axiom, pred, section, local, match, with}
}

\renewcommand*{\lstlistingname}{Code Example}

% Define easycrypt environment
\lstnewenvironment{easycrypt}[1][]
{
	\noindent
	\minipage{\linewidth}
	\vspace{0.5\baselineskip}
	\lstset{ % Set inline code parameter(s)
	  basicstyle=\small,						% print whole listing small
	  keywordstyle=\color{black}\bfseries,		% bold black keywords
	  identifierstyle=, 						% nothing happens
	  stringstyle=\ttfamily, 					% typewriter type for strings
	  showstringspaces=false,					% no special string spaces
	  tabsize=2,								% tabs typeset as 2 spaces
	  frame=single,								% lines above and below code snippets
	  breaklines=true,							% break lines when there's overflow
	  language=easycrypt,
	  morecomment=[n][\color{gray}]{(*}{*)},	% define normal comment delimiters, prints gray
	  escapeinside={/*}{*/},					% special delimiters for referencing line numbers
	  literate={~} {$\sim$}{1},					% prettier tilde
	  captionpos=b,								% caption position is bottom
	  numbers=left,								% print line numbers on the left
	  #1										% other custom parameters
	}
}{\endminipage}

% % help cref
\crefname{listing}{Code example}{Code examples} % upper case, personal preference
\Crefname{listing}{Code example}{Code examples}

\newcommand{\code}[1]{\texttt{#1}} % inline code

% == Math ==
% operators
\newcommand{\Or}{\vee}
\renewcommand{\And}{\wedge}
\DeclareMathOperator{\Concat}{\; || \; }
\newcommand{\Union}{\cup}
\newcommand{\LDE}[1]{\hat{#1}}
\newcommand{\RLDE}[1]{\bar{#1}}
\newcommand{\iseq}{\overset{?}{=}}
\newcommand{\eqdef}{\ {:=} \ }
\newcommand{\ip}[2]{\langle #1 , #2 \rangle}
\newcommand{\Time}[1]{\mathrm{time}(#1)}


% == Messages ==
\newcommand{\OpenMsg}{\mathsf{Open}}
\newcommand{\CommitMsg}{\mathsf{Commit}}
\newcommand{\Input}{x}

\newcommand{\SecParam}{\lambda}

% PKE
\newcommand{\PKE}{\mathsf{PKE}}
\newcommand{\Gen}{\mathsf{Gen}}
\newcommand{\Enc}{\mathsf{Enc}}
\newcommand{\OblivEnc}{\mathsf{OblivEnc}}
\newcommand{\OblivEncInv}{\mathsf{OblivEncInv}}
\newcommand{\Dec}{\mathsf{Dec}}
\newcommand{\Indcpa}{\mathsf{INDCPA}}
\newcommand{\IndcpaGM}{\mathcal{C}} % the game master (i.e. challenger) for the indcpa game.
\newcommand{\IndcpaAdversary}{{\Adversary_\Indcpa}}
\newcommand{\EncKey}{\mathsf{pk}}
\newcommand{\DecKey}{\mathsf{sk}}
\newcommand{\GuessBit}{b'}
\newcommand{\ChallengerBit}{b}
\newcommand{\EncOracle}{\mathsf{EO}}
\newcommand{\DecOracle}{\mathsf{DO}}
\newcommand{\Msg}{m}
\newcommand{\Ciphertext}{c}
\newcommand{\Rand}{r}
\newcommand{\FakeRand}{r_{\mathsf{fake}}}

% CFPTP
\newcommand{\CFPTP}{\mathsf{CFPTP}}
\newcommand{\Forw}{\mathsf{Forw}}
\newcommand{\Back}{\mathsf{Back}}
\newcommand{\ForwKey}{\mathsf{fk}}
\newcommand{\BackKey}{\mathsf{bk}}
\newcommand{\Domain}{D}
\newcommand{\CFPTPAdversary}{{\Adversary_\CFPTP}}
\newcommand{\CFPTPInput}{\Input}
\newcommand{\CFPTPOutput}{y}
\newcommand{\CFPTPBit}{b}

% == Proof ==
% UC
\newcommand{\Simulator}{{\mathsf{Sim}}} % Simulator
\newcommand{\Adversary}{{\mathsf{Adv}}} % Adversary
\newcommand{\AdversaryB}{{\mathsf{B}}} % Adversary
\newcommand{\Environment}{{\mathcal{Z}}} % Environment
\newcommand{\IF}{\mathcal{F}} % Ideal functionality
\newcommand{\sid}{\mathsf{sid}}

% Hybrids
\newcommand{\Ideal}{{\mathsf{IDEAL}}}
\newcommand{\Hyb}{{\mathsf{HYB}}}
\newcommand{\Real}{{\mathsf{REAL}}}
\newcommand{\bad}{{\mathsf{bad}}}
\newcommand{\True}{\mathsf{true}}
\newcommand{\False}{\mathsf{false}}
\newcommand{\G}{\mathsf{G}}

\newcommand{\pST}{\; \middle| \;}
\newcommand{\poly}{\mathrm{poly}}
\newcommand{\negl}{\mathrm{negl}}
\newcommand{\Bits}{\{0,1\}}
\newcommand{\Naturals}{\mathbb{N}}
\newcommand{\Reals}{\mathbb{R}}
\newcommand{\Integers}{\mathbb{Z}}

% CRS
\newcommand{\IFCrs}{\IF_{\mathrm{CRS}}}
\newcommand{\CrsDistr}{\mathcal{D}}

% Commitment
\newcommand{\IFCom}{\IF_{\mathrm{COM}}}
\newcommand{\IFMcom}{\IF_{\mathrm{MCOM}}}
\newcommand{\Party}{\mathrm{P}}
\newcommand{\PartyCommitter}{\mathrm{C}}
\newcommand{\PartyVerifier}{\mathrm{V}}
\newcommand{\CommittedBit}{b}

% Files
\newcommand{\File}[1]{\texttt{#1}}

% == Document title ==
\title{\cite{CanettiF01} adaptive security proof}
\author{Megan Chen}
\date{\today}

\begin{document}
\maketitle
\tableofcontents

%%%%%%%%%%%%%%%%%%%%%%%%%%%%%%%%%%%%%%%%%%%%%%%%%%%%%%%%%%%%%%%%%%%%%%%%%%%%%%%%
%%%%%%%%%%%%%%%%%%%%%%%%%%%%%%%%%%%%%%%%%%%%%%%%%%%%%%%%%%%%%%%%%%%%%%%%%%%%%%%%
%%%%%%%%%%%%%%%%%%%%%%%%%%%%%%%%%%%%%%%%%%%%%%%%%%%%%%%%%%%%%%%%%%%%%%%%%%%%%%%%
%%%%%%%%%%%%%%%%%%%%%%%%%%%%%%%%%%%%%%%%%%%%%%%%%%%%%%%%%%%%%%%%%%%%%%%%%%%%%%%%
\doclearpage
\section{Preliminaries}
\label{sec:preliminaries}

\subsection{Claw-free pair of trapdoor permutations (CFPTP)}
A \emph{claw-free pair of trapdoor permutations} over a domain $\Domain$ is defined as $\CFPTP = (\Gen, \Forw, \Back, \Dec)$, such that:
\begin{itemize}
	\item $\Gen(1^\SecParam)$: On input a security parameter $1^\SecParam$, outputs a forward key $\ForwKey$ and backward key $\BackKey$.
	\item $\Forw_{\CFPTPBit, \ForwKey}(\CFPTPInput)$: On input a bit $\CFPTPBit$, a forward key $\ForwKey$ and a value $\Input \in \Domain$, outputs the result $\CFPTPOutput \in \Domain$ of running permutation $\CFPTPBit$ forwards.
	\item $\Back_{\CFPTPBit, \BackKey}(\CFPTPOutput)$: On input a bit $\CFPTPBit$, a backward key $\BackKey$ and a value $\CFPTPOutput \in \Domain$, outputs the result $\Input \in \Domain$ of running permutation $b$ backwards.
\end{itemize}
We require that $\CFPTP$ is claw-free: given only $\ForwKey$, it is hard to find $\CFPTPInput_0, \CFPTPInput_1$ such that $\Forw_{\CFPTPBit, \ForwKey}(\CFPTPInput_0) = \Forw_{\CFPTPBit, \ForwKey}(\CFPTPInput_1)$

\subsection{Public key encryption (PKE)}

A \emph{public key encryption scheme} $\PKE = (\Gen, \Enc, \Dec)$ works as follows:
\begin{itemize}
	\item $\Gen(1^\SecParam)$: On input a security parameter $1^\SecParam$, outputs an encryption key $\EncKey$ and decryption key $\DecKey$.
	\item $\Enc(\EncKey,\Msg)$: On input an encryption key $\EncKey$ and message $\Msg$, outputs an encryption of message $\Msg$.
	\item $\Dec_{\DecKey}(\Ciphertext)$: On input a decryption key $\DecKey$ and ciphertext $\Ciphertext$, outputs the decrypton of a ciphertext $\Ciphertext$.
\end{itemize}

$\PKE$ satisfies the following properties:
\begin{itemize}
    \item \emph{Correctness.} For all keys $(\EncKey, \DecKey) \gets \Gen(1^{\SecParam})$ and messages $\Msg$, we have
    \[ \Pr[ \Dec_{\DecKey} (\Enc_{\EncKey} (\Msg)) = \Msg ] = 1. \]

    \item \emph{Obliviously-sampleable.} There additionally exist algorithms $(\OblivEnc, \OblivEncInv)$ with the following syntax:
    \begin{itemize}
    	\item $\OblivEnc(\EncKey,\Rand)$: On input the encryption key $\EncKey$ and randomness $\Rand$, outputs a ciphertext $\Ciphertext$.
    	\item $\OblivEncInv(\EncKey,\Ciphertext)$: On input the encryption key $\EncKey$ and a ciphertext $\Ciphertext$, outputs randomness $\FakeRand$.
    \end{itemize}

    Further, the following distributions are negligibly-close in statistical distance:
    \newcommand{\Distribution}{\mathcal{D}}
    \begin{align*}
		&\Distribution_0(\SecParam) \eqdef \left\{
		\Adversary(\Ciphertext,\Rand)
		\pST
		\begin{array}{r}
            (\EncKey,\DecKey)\gets \Gen(1^\SecParam)\\
			\Msg \gets \Adversary(\EncKey)\\
            \Rand \gets \Bits^{*}\\
			\Ciphertext \gets \OblivEnc(\EncKey,\vec{0})\\
		\end{array}
		\right\} \\
		&\text{and } \Distribution_1(\SecParam) \eqdef \left\{
		\Adversary(\Ciphertext,\Rand_1)
		\pST
		\begin{array}{r}
            (\EncKey,\DecKey)\gets \Gen(1^\SecParam)\\
            \Msg \gets \Adversary(\EncKey)\\
            \Rand_0 \gets \Bits^{*}\\
            \Ciphertext \gets \Enc(\EncKey,\Msg;\Rand_0)\\
			\Rand_1 \gets \OblivEncInv(\EncKey,\Ciphertext)\\
		\end{array}
		\right\} \enspace.
	\end{align*}
\end{itemize}

\begin{lemma}
    Let $\PKE = (\Gen, \Enc, \Dec, \OblivEnc, \OblivEncInv)$ be an obliviously-sampleable CCA-secure public key encryption scheme and $\SecParam\in\Naturals$ be a security parameter. Then for all adversaries $\Adversary$,
    \begin{equation}
    \label{eqn:prod-msg-rand}
        \Pr \left[
            \Enc(\EncKey,\Msg';\Rand') = \Ciphertext
        \pST
        \begin{array}{r}
            (\EncKey,\DecKey) \gets \Gen(1^{\SecParam})\\
            \Rand \gets \Bits^*\\
            \Ciphertext \gets \OblivEnc(\EncKey, \Rand)\\
            (\Msg',\Rand') \gets \Adversary(\Ciphertext)\\
        \end{array}
        \right] = \negl(\SecParam).
    \end{equation}
\end{lemma}

\begin{proof}
    Suppose there exists an adversary $\Adversary$ who satisfies the probability in \Cref{eqn:prod-msg-rand} with probability $\delta$. We argue via reduction that this implies breaking the oblivious-sampleability property of $\PKE$.

    Construct a (stateful) adversary $\AdversaryB$ that breaks the oblivious-sampleabilty of $\PKE$ as follows:
    \begin{itemize}
        \item \textbf{Message generation.} $\AdversaryB(\EncKey) \to \Msg$
        \begin{enumerate}[nolistsep]
            \item Parse $\EncKey$ as the encryption key.
            \item Output a random $\Msg$.
        \end{enumerate}
        \item \textbf{Message generation.} $\AdversaryB(\Ciphertext,\Rand) \to \GuessBit$
        \begin{enumerate}[nolistsep]
            \item Run $(\Msg',\Rand')\gets \Adversary(\Ciphertext)$.
            \item Output $\GuessBit$.
        \end{enumerate}
    \end{itemize}
    By the correctness of $\Adversary$, we have $\Enc(\EncKey, \Msg'; \Rand') = \Ciphertext$.
    \megan{I'm unsure how to finish analyzing the reduction. This is because, in $\Distribution_0$ and $\Distribution_1$, either $\Ciphertext$ or $\Rand$ is fake, and it's unclear how to tell the difference using $\Adversary$}
\end{proof}

\subsection{Adaptive chosen-ciphertext attack security (LR-CCA2)}

\cite{CanettiF01} uses a left-or-right oracle formulation of security against chosen ciphertext attacks. Let $\SecParam\in\Naturals$ be a security parameter. First, define an encryption-decryption oracle pair $(\EncOracle, \DecOracle)$ with respect to a key pair $(\EncKey,\DecKey) \gets \Gen(1^\SecParam)$ as follows:
\begin{itemize}
	\item Encryption oracle

	\begin{minipage}{0.9\textwidth}
	$\EncOracle(\ChallengerBit,\EncKey):$
	\begin{enumerate}[nolistsep]
		\item \textbf{Init:} Receive a bit $\ChallengerBit$ and an encryption key $\EncKey$.
		\item \textbf{Encryption queries:} Upon receiving a message pair $(\Msg_0, \Msg_1)$, output $\Enc(\EncKey,\Msg_{\ChallengerBit})$.
	\end{enumerate}
	\end{minipage}

	\item Decryption oracle

	\begin{minipage}{0.9\textwidth}
	$\DecOracle(\DecKey):$
	\begin{enumerate}[nolistsep]
		\item \textbf{Init:} Receive a decryption key $\DecKey$.
		\item \textbf{Decryption queries:} Upon receiving a ciphertext $\Ciphertext$, output $\Dec_{\DecKey}(\Ciphertext)$.
	\end{enumerate}
	\end{minipage}
\end{itemize}


A public key encryption scheme $\PKE$ is secure against left-or-right adaptive chosen ciphertext attacks (LR-CCA2) if for any PPT adversary $\Adversary$:

\begin{align*}
	\Pr \left[
	\begin{array}{c}
		\GuessBit = \ChallengerBit
	\end{array}
	\pST
	\begin{array}{r}
		(\EncKey, \DecKey) \gets \Gen(1^{\SecParam})\\
		\ChallengerBit \gets \Bits\\
		\GuessBit \gets \Adversary^{\EncOracle(\ChallengerBit,\EncKey), \DecOracle(\DecKey)}(\EncKey)\\
	\end{array}
	\right] \leq \frac{1}{2} + \frac{1}{\poly(\SecParam)}.
\end{align*}

%%%%%%%%%%%%%%%%%%%%%%%%%%%%%%%%%%%%%%%%%%%%%%%%%%%%%%%%%%%%%%%%%%%%%%%%%%%%%%%%
%%%%%%%%%%%%%%%%%%%%%%%%%%%%%%%%%%%%%%%%%%%%%%%%%%%%%%%%%%%%%%%%%%%%%%%%%%%%%%%%
%%%%%%%%%%%%%%%%%%%%%%%%%%%%%%%%%%%%%%%%%%%%%%%%%%%%%%%%%%%%%%%%%%%%%%%%%%%%%%%%
%%%%%%%%%%%%%%%%%%%%%%%%%%%%%%%%%%%%%%%%%%%%%%%%%%%%%%%%%%%%%%%%%%%%%%%%%%%%%%%%
\doclearpage
\section{Ideal functionalities}
\label{sec:ideal-functionalities}

%%%%%%%%%%%%%%%%%%%%%%%%%%%%%%%%%%%%%%%%%%%%%%%%%%%%%%%%%%%%%%%%%%%%%%%%%%%%%%%%
%%%%%%%%%%%%%%%%%%%%%%%%%%%%%%%%%%%%%%%%%%%%%%%%%%%%%%%%%%%%%%%%%%%%%%%%%%%%%%%%
\subsection{CRS}
\label{sec:ideal-func-crs}
$\IFCrs$ is exactly as in Figure 1 of \cite{CanettiF01}, except that the CRS distribution $\CrsDistr$ is sampled as follows:
\begin{enumerate}[nolistsep]
	\item $(\ForwKey,\BackKey) \gets \CFPTP.\Gen(1^{\SecParam})$.
	\item $(\EncKey,\DecKey) \gets \PKE.\Gen(1^{\SecParam})$.
	\item Set the CRS value $d \eqdef (\ForwKey, \EncKey)$.
\end{enumerate}

%%%%%%%%%%%%%%%%%%%%%%%%%%%%%%%%%%%%%%%%%%%%%%%%%%%%%%%%%%%%%%%%%%%%%%%%%%%%%%%%
%%%%%%%%%%%%%%%%%%%%%%%%%%%%%%%%%%%%%%%%%%%%%%%%%%%%%%%%%%%%%%%%%%%%%%%%%%%%%%%%
\subsection{Single-session commitment, adaptive}
\label{sec:ideal-func-com}

We define an ideal functionality for single-session commitments that allows adaptive corruptions and mirrors what is in \File{Commitment.uc}.
\newcommand{\EnvCommitReqMsg}{\mathsf{EnvCommitReq}}
\newcommand{\IFCommitReqMsg}{\mathsf{IFCommitReq}}
\newcommand{\SimCommitterCorruptionStatusMsg}{\mathsf{CommitterCorruptionStatus}}
\newcommand{\CommitterBitMsg}{\mathsf{CommitterBit}}
\newcommand{\CommitterCorruptedBit}{\mathsf{corrupted}}
\newcommand{\CommitOkMsg}{\mathsf{CommitOk}}

\begin{center}
\begin{minipage}{0.9\textwidth}
$\IFCom:$
\begin{enumerate}
	\item \textbf{Activation.} Receive $(\EnvCommitReqMsg, \sid, \PartyCommitter, \PartyVerifier, \CommittedBit)$ from $\Environment$, and send $(\IFCommitReqMsg, \sid, \PartyCommitter,\PartyVerifier)$ to $\Simulator$.

	\item \textbf{Initial corruption sequence.} Receive $(\SimCommitterCorruptionStatusMsg, \CommitterCorruptedBit)$ from $\Simulator$.
	\begin{itemize}
		\item If $\CommitterCorruptedBit = \True$, send $(\CommitterBitMsg, \CommittedBit)$ to $\Simulator$.
		\item If $\CommitterCorruptedBit = \False$, send $(\CommitterBitMsg, \bot)$ to $\Simulator$.
	\end{itemize}

	\item \textbf{Commit.} Receive $(\CommitOkMsg)$ from $\Simulator$ and send $(\CommitMsg, \sid, \PartyCommitter, \PartyVerifier)$ to $\PartyVerifier$.

	\item \textbf{Open.} \megan{TODO}

	\item \textbf{Corruption.} Receive $(\EnvCommitReqMsg, \sid, \PartyCommitter, \PartyVerifier, \CommittedBit)$ from $\Environment$. Send $(\IFCommitReqMsg, \sid, \PartyCommitter,\PartyVerifier, \CommittedBit)$ to $\Simulator$.
\end{enumerate}
\end{minipage}
\end{center}

%%%%%%%%%%%%%%%%%%%%%%%%%%%%%%%%%%%%%%%%%%%%%%%%%%%%%%%%%%%%%%%%%%%%%%%%%%%%%%%%
%%%%%%%%%%%%%%%%%%%%%%%%%%%%%%%%%%%%%%%%%%%%%%%%%%%%%%%%%%%%%%%%%%%%%%%%%%%%%%%%
%%%%%%%%%%%%%%%%%%%%%%%%%%%%%%%%%%%%%%%%%%%%%%%%%%%%%%%%%%%%%%%%%%%%%%%%%%%%%%%%
%%%%%%%%%%%%%%%%%%%%%%%%%%%%%%%%%%%%%%%%%%%%%%%%%%%%%%%%%%%%%%%%%%%%%%%%%%%%%%%%
\doclearpage
\section{Scenario}
\label{sec:scenario}

The $\Environment$/$\Adversary$'s job is to distinguish the real and ideal distributions.

%%%%%%%%%%%%%%%%%%%%%%%%%%%%%%%%%%%%%%%%%%%%%%%%%%%%%%%%%%%%%%%%%%%%%%%%%%%%%%%%
%%%%%%%%%%%%%%%%%%%%%%%%%%%%%%%%%%%%%%%%%%%%%%%%%%%%%%%%%%%%%%%%%%%%%%%%%%%%%%%%
\subsection{Real Distribution: committer is adaptively corrupted}
\label{sec:real-distribution}

\begin{minipage}{0.9\textwidth}
\parhead{Real distribution} The committer does the following:
\begin{enumerate}[nolistsep]
	\item \parhead{Commit round}

	\begin{enumerate}[nolistsep]
		\item Receive $(\CommittedBit,\sid)$ from $\Environment$.
		\item Send $\IFCrs$
		\item Sample $\Input_{\CommittedBit}$ from the plaintext space.
		\item Sample $\Rand_{\CommittedBit}$ and $\Rand_{1-\CommittedBit}$ from the commitment randomness space.
		\item Compute $\CFPTPOutput = \CFPTP.\Forw_{\CommittedBit, \ForwKey}(\Input_{\CommittedBit})$.
		\item Compute $\Ciphertext_{\CommittedBit} = \PKE.\Enc_\EncKey(\Input, \sid; \Rand_\CommittedBit)$.
		\item Compute $\Ciphertext_{1-\CommittedBit} = \PKE.\OblivEnc_\EncKey(\Rand_{1-\CommittedBit})$.
		\item Send $(\CFPTPOutput, \Ciphertext_0, \Ciphertext_1)$ to $\Adversary$.
	\end{enumerate}

	\item \parhead{Open round}

	\begin{enumerate}[nolistsep]
		\item tbd
	\end{enumerate}
\end{enumerate}
\end{minipage}

%%%%%%%%%%%%%%%%%%%%%%%%%%%%%%%%%%%%%%%%%%%%%%%%%%%%%%%%%%%%%%%%%%%%%%%%%%%%%%%%
%%%%%%%%%%%%%%%%%%%%%%%%%%%%%%%%%%%%%%%%%%%%%%%%%%%%%%%%%%%%%%%%%%%%%%%%%%%%%%%%
\subsection{Ideal Distribution}
\label{sec:ideal-distribution}

\begin{minipage}{0.9\textwidth}
\parhead{Ideal distribution} $\Simulator$ does the following:
\begin{enumerate}[nolistsep]
	\item \parhead{Commit round}
	\begin{enumerate}[nolistsep]
		\item Receive $\sid$ from $\Environment$
		\item Sample $\CFPTPOutput$ from the plaintext space.
		\item Sample $\Rand_{0}$ and $\Rand_{1}$ from the commitment randomness space.
		\item Compute $\Ciphertext_{0} = \PKE.\Enc(\EncKey,\Input_{0}, \sid; \Rand_{0})$.
		\item Compute $\Ciphertext_{1} = \PKE.\Enc(\EncKey,\Input_{1}, \sid; \Rand_{1})$.
		\item Send $(\CFPTPOutput, \Ciphertext_0, \Ciphertext_1)$ to $\Adversary$.
	\end{enumerate}

	\item \parhead{Open round}

	\begin{enumerate}[nolistsep]
		\item Receive from $\IF$ that the committed bit is $\CommittedBit$.
	\end{enumerate}

\end{enumerate}
\end{minipage}

%%%%%%%%%%%%%%%%%%%%%%%%%%%%%%%%%%%%%%%%%%%%%%%%%%%%%%%%%%%%%%%%%%%%%%%%%%%%%%%%
%%%%%%%%%%%%%%%%%%%%%%%%%%%%%%%%%%%%%%%%%%%%%%%%%%%%%%%%%%%%%%%%%%%%%%%%%%%%%%%%
%%%%%%%%%%%%%%%%%%%%%%%%%%%%%%%%%%%%%%%%%%%%%%%%%%%%%%%%%%%%%%%%%%%%%%%%%%%%%%%%
%%%%%%%%%%%%%%%%%%%%%%%%%%%%%%%%%%%%%%%%%%%%%%%%%%%%%%%%%%%%%%%%%%%%%%%%%%%%%%%%
\doclearpage
{\small{
\bibliographystyle{alpha}
\bibliography{refs}
}}

\end{document}
