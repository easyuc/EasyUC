\def\IsDraft{} % set for draft version

\documentclass{article}[12pt]

% == Import Packages ==
\usepackage[margin=1in]{geometry} % Page dimensions
\usepackage{amsmath} % Math formulas
\usepackage{xcolor} % Colors
\usepackage{listings} % Listings, for code snippets
\usepackage{hyperref} % Hyperlinks
\usepackage[capitalise]{cleveref} % Clever References, must be loaded AFTER all other packages

% == Formatting ==
\newcommand{\parhead}[1]{\textbf{#1}~}
\renewcommand{\emph}[1]{\textbf{#1}~}

% == Inline comments ==
\ifdefined\IsDraft
\newcommand{\authnote}[2]{[{\color{red}\textbf{#1:}}~{\color{blue} #2}]}
\else
\newcommand{\authnote}[2]{}
\fi

\newcommand{\alley}[1]{\authnote{Alley}{#1}}
\newcommand{\megan}[1]{\authnote{Megan}{#1}}
\newcommand{\mayank}[1]{\authnote{Mayank}{#1}}
\newcommand{\ran}[1]{\authnote{Ran}{#1}}

% == Code snippets ==
% Listings package docs: http://texdoc.net/texmf-dist/doc/latex/listings/listings.pdf

\lstdefinelanguage{easycrypt}{
	morekeywords={lemma, module, proc, var, return, if, else, require, import, type, op, axiom, pred, section, local, match, with}
}

\renewcommand*{\lstlistingname}{Code Example}

% Define easycrypt environment
\lstnewenvironment{easycrypt}[1][]
{
	\noindent
	\minipage{\linewidth}
	\vspace{0.5\baselineskip}
	\lstset{ % Set inline code parameter(s)
	  basicstyle=\small,						% print whole listing small
	  keywordstyle=\color{black}\bfseries,		% bold black keywords
	  identifierstyle=, 						% nothing happens
	  stringstyle=\ttfamily, 					% typewriter type for strings
	  showstringspaces=false,					% no special string spaces
	  tabsize=2,								% tabs typeset as 2 spaces
	  frame=single,								% lines above and below code snippets
	  breaklines=true,							% break lines when there's overflow
	  language=easycrypt,
	  morecomment=[n][\color{gray}]{(*}{*)},	% define normal comment delimiters, prints gray
	  escapeinside={/*}{*/},					% special delimiters for referencing line numbers
	  literate={~} {$\sim$}{1},					% prettier tilde
	  captionpos=b,								% caption position is bottom
	  numbers=left,								% print line numbers on the left
	  #1										% other custom parameters
	}
}{\endminipage}

% % help cref
\crefname{listing}{Code example}{Code examples} % upper case, personal preference
\Crefname{listing}{Code example}{Code examples}

\newcommand{\code}[1]{\texttt{#1}} % inline code

% == Messages ==
\newcommand{\OpenMsg}{\mathsf{Open}}
\newcommand{\CommitMsg}{\mathsf{Commit}}

\newcommand{\SecParam}{\lambda}
% PKE
\newcommand{\PKE}{\mathsf{PKE}}
\newcommand{\Gen}{\mathsf{Gen}}
\newcommand{\Enc}{\mathsf{Enc}}
\newcommand{\Dec}{\mathsf{Dec}}
\newcommand{\Indcpa}{\mathsf{INDCPA}}
\newcommand{\IndcpaGM}{\mathcal{C}} % the game master (i.e. challenger) for the indcpa game.
\newcommand{\EncKey}{\mathsf{pk}}
\newcommand{\DecKey}{\mathsf{sk}}

% CFPTP
\newcommand{\CFPTP}{\mathsf{CFPTP}}
\newcommand{\Forw}{\mathsf{Forw}}
\newcommand{\Back}{\mathsf{Back}}
\newcommand{\ForwKey}{\mathsf{fk}}
\newcommand{\BackKey}{\mathsf{bk}}
\newcommand{\Domain}{D}


% == Proof ==
% Entities
\newcommand{\Simulator}{{\mathsf{Sim}}} % Simulator
\newcommand{\Adversary}{{\mathsf{Adv}}} % Adversary
\newcommand{\Environment}{{\mathcal{Z}}} % Environment
\newcommand{\CFPTPAdversary}{{\Adversary_\CFPTP}}
\newcommand{\IndcpaAdversary}{{\Adversary_\Indcpa}}

% Hybrids
\newcommand{\Ideal}{{\mathsf{IDEAL}}}
\newcommand{\Hyb}{{\mathsf{HYB}}}
\newcommand{\Real}{{\mathsf{REAL}}}
\newcommand{\bad}{{\mathsf{bad}}}
\newcommand{\True}{1}
\newcommand{\G}{\mathsf{G}}


% == Document title ==
\title{Sketch for the \cite{CanettiF01} static protocol's formal proof}
\author{Megan Chen}
\date{\today}

\begin{document}
\maketitle
\tableofcontents

\section{Preliminaries}

\subsection{Claw-free pair of trapdoor permutations (CFPTP)}
A \emph{claw-free pair of trapdoor permutations} over a domain $\Domain$ is defined as $\CFPTP = (\Gen, \Forw, \Back, \Dec)$, such that:
\begin{itemize}
	\item $\Gen(1^\SecParam)$: Outputs a forward key $\ForwKey$ and backward key $\BackKey$.
	\item $\Forw_{b, \ForwKey}(x)$: Outputs the result $y\in \Domain$ of running permutation $b$ forwards.
	\item $\Back_{b, \BackKey}(y)$: Outputs the result $x\in \Domain$ of running permutation $b$ backwards.
\end{itemize}
We require that $\CFPTP$ is claw-free: given only $\ForwKey$, it is hard to find $x_0, x_1$ such that $\Forw_{b, \ForwKey}(x_0) = \Forw_{b, \ForwKey}(x_1)$


\subsection{Public key encryption (PKE)}

A \emph{public key encryption scheme} $\PKE = (\Gen, \Enc, \Dec)$ works as follows:
\begin{itemize}
	\item $\Gen(1^\SecParam)$: Outputs an encryption key $\EncKey$ and decryption key $\DecKey$.
	\item $\Enc_{\EncKey}(m)$: Outputs an encryption of message $m$.
	\item $\Dec_{\DecKey}(c)$: Outputs the decrypton of a ciphertext $c$.
\end{itemize}
We require that $\PKE$ is IND-CPA (i.e. semantically) secure.

\section{Proof strategy overview}

The proof of indistinguishability between the $\Real$ and $\Ideal$ executions proceeds via a sequence of hybrid games. First, we define the hybrid execution $\Hyb$, which is an intermediate step between $\Real$ and $\Ideal$.

\subsection{Defining the hybrid execution: $\Hyb$}

$\Hyb$ to works exactly like the $\Ideal$ execution except for the following case: If the committer is \underline{not} corrupted, the simulator $\Simulator_{\Hyb}$ learns the real committed bit $b$. This is because, like in $\Ideal$, $\Simulator_{\Hyb}$ generates the CRS string using both the CFPTP and PKE's key generation algorithms. Hence, $\Simulator_{\Hyb}$ has access to the CFPTP's backward key $\BackKey$ and the PKE scheme's secret (or decryption) key $\DecKey$. Thus, upon seeing the committer's message $\CommitMsg = (y, c_0, c_1)$, $\Simulator_{\Hyb}$ can recover $b$ because it can decrypt and run the permutation backwards.

We detail the difference between how $(y, c_0, c_1)$ is computed in $\Hyb$ and in $\Ideal$. In $\Hyb$, the committer's $\CommitMsg$ message will be the string $(y, c_0, c_1)$ that is generated as in the real protocol, i.e. choose a random $x \in D$, then compute $y = \CFPTP.\Forw_{b, \ForwKey}(x)$, $c_b = \PKE.\Enc_\EncKey(x; r_b)$, and $c_{1-b} = \PKE.\Enc_\EncKey(0; r_{1-b})$ such that $r_0, r_1$ are uniformly sampled from the randomness distribution.

Contrastingly in the $\Ideal$ when the committer is honest, $\Simulator$'s $\CommitMsg$ message is $(y, c_0, c_1)$ where $y$ is first selected from $\Domain$, then its preimages are computed: $x_0 = \CFPTP.\Back_{0, \BackKey}(y)$ and $x_1 = \CFPTP.\Back_{1, \BackKey}(y)$. Subsequently, the ciphertexts are computed: $c_0 = \Enc_\EncKey(x_0; r_0)$ and $c_1 = \Enc_\EncKey(x_1; r_1)$ such that $r_0, r_1$ are uniformly sampled from the randomness distribution.

\parhead{$\Hyb$'s bad event.}
Due to $\Hyb$'s definition, the environment $\Environment$'s view between $\Real$ and $\Hyb$ differs when $\Hyb$'s simulator $\Simulator_\Hyb$ aborts. Specifically, $\Simulator_\Hyb$ aborts when it receives a $\CommitMsg$ message $(y, c_0, c_1)$ and an $\OpenMsg$ message $(b', x', r)$ such that
\begin{itemize}
	\item $x' = \CFPTP.\Back_{b', \BackKey}(y)$
	\item $c_{b'} = \PKE.\Enc_\EncKey(x'; r)$
	\item $b' \ne b$, where $b$ is the bit $\Simulator_\Hyb$ detects from the $\CommitMsg$ message.
\end{itemize}

This is a ``bad event'' because $\Simulator_\Hyb$ found a claw for the pair of trapdoor permutations. Recall that in $\Hyb$, if $b$ is the bit that $\Simulator_\Hyb$ detects from the $\CommitMsg$ message, then $\CFPTP.\Back_{b, \BackKey}(y) = x$. Hence, the claw is $x, x'$.

\subsection{Sequence of Hybrids}
The proof proceeed via a sequence of hybrids. The two main hybrids are:
\begin{enumerate}
	\item\label{hyb:1} $\Real \approx \Hyb$ - This is proved via a reduction to the claw-free property of the CFPTP scheme. \cref{sec:hyb1} explains the formal proof.
	\item\label{hyb:2} $\Hyb \approx \Ideal$ - This is proved via a reduction to the IND-CPA security of the PKE scheme. \cref{sec:hyb1} explains the formal proof.
\end{enumerate}

\subsection{The proof's goal lemma}
The overall goal is the show that the ability to distinguish between $\Real$ and $\Hyb$ is bounded by the probability of finding a claw in the CFPTP scheme, summed with the probability of distinguishing ciphertexts in the PKE scheme.

\begin{easycrypt}[label=code:lem_real_ideal, caption=Main lemma for indistinguishability between $\Real$ and $\Ideal$ views]
lemma REAL_IDEAL &m :
`|Pr[REAL().main() @ &m : res] - Pr[Ideal().main() @ &m : res]|
<= Pr[CFP_Game(CFAdv).main() @ &m : res]
   + `|Pr[INDCPA_0(INDCPAAdv).main() @ &m : res] - Pr[INDCPA_1(INDCPAAdv).main() @ &m : res]|.
\end{easycrypt}

\section{Hybrid \ref{hyb:1}: Showing $\Real \approx \Hyb$}\label{sec:hyb1}

\subsection{Mini Hybrids}
To accomplish the reduction to claw-free property of the CFPTP scheme, we require the following hybrid steps:
\begin{enumerate}
	\item\label{hyb:1.1} $\Real \approx \Real'$ - Update $\Real$ to include the definition of a bad event, which occurs if and only if the bad event in $\Hyb$ occurs. Call this updated execution $\Real'$. In $\Real'$, we add a global variable \code{REAL'.bad}, initialize it to false, and flip it to true when the bad event occurs.

	The use of easycrypt's ``up to bad reasoning'' step motivates this switch; typically, doing ``up to bad reasoning'' requires a bad event to occur in both programs, whereas the bad event doesn't occur in $\Real$ because $\Simulator$ cannot detect it. This ensures

	Ultimately, easycrypt's \code{byequiv} tactic will require proving

\begin{easycrypt}[label=code:up_to_bad_claim, caption=Claim required to utilize ``up to bad reasoning'']
equiv [REAL'.main ~ HYB.main : true ==> (REAL'.bad{1} <=> HYB.bad{2}) /\ (! HYB.bad{2} => ={res})]
\end{easycrypt}

	where \code{REAL'.bad} and \code{HYB.bad} are global Boolean variables initialized to false and set to true when \code{REAL'} and \code{HYB}'s respective bad events occur. In other words, the proof requires showing that after running the two programs, \code{REAL'.bad} and \code{HYB.bad} take the same value in \code{REAL'} and \code{HYB}'s respective memories. Also if \code{HYB.bad} (thus also \code{REAL'.bad}) is false, then the two program executions are equal.

	To properly detect the bad event, $\Real'$'s simulator needs access to the CFPTP's backward key $\BackKey$ and the PKE's decryption key $\DecKey$, which are not part of $\Real$'s CRS string. To make $\BackKey, \DecKey$ accessible, we add them to the CRS string and save the new CRS as a global variable, so all parties (committer, simulator, verifier) may access it.

	Also, note that we move the CRS generation step to be at the beginning of $\Real'$, so the committer can access it. This shift doesn't affect $\Real'$'s output, since the CRS is still sampled the same way using the CFPTP and PKE key generation algorithms. \megan{This can be a separate hybrid step, or keep it as part of $\Real'$.}

	$\Real'$'s executions works as follows: when the committer learns that it's corrupted, it does what $\Hyb$'s simulator does with $(y, c_0, c_1)$, using $\CFPTP.\Back_0, \CFPTP.\Back_1$ and $\PKE.\Dec$ to figure out what $b$ is. The simulator stores this $b$, but does nothing else with it. When the verifier is given $(b', x, r)$, it does the verification step like $\Real$ does.

	If the verifier accepts but $b'$ is not equal to the $b$ that the committer saved, then the verifier sets \code{REAL'.bad} to true. This corresponds with the simulator in $\Hyb$ setting \code{HYB.bad} to true and aborting.

	Note that in $\Real'$, the $\OpenMsg$ message is still sent to \code{pt2} as usual. This is crucial for retaining the program equivalence of $\Real$ and $\Real'$.

	\item\label{hyb:1.2} $\Real' \approx \Hyb$ - Use ``up to bad reasoning'' to show the reduction to breaking the CFPTP scheme. That is, distinguishing $\Real'$ and $\Hyb$ implies building an adversary that can find a claw in the CFPTP scheme. The details are in \cref{sec:cfptp_overview,sec:cfptp_easycrypt}.

\end{enumerate}

\subsection{Overview}\label{sec:cfptp_overview}
We use up to bad reasoning to show an upper bound on the environment $\Environment$'s ability to distinguish between $\Real'$ and $\Hyb$.

Hence, we upper-bound $\Environment$'s distinguishing advantage via defining an adversary $\CFPTPAdversary$ that finds claws for a CFPTP, i.e. the bad event has occurred. $\CFPTPAdversary$'s probability of finding a claw upper bounds $\Environment$'s distinguishing advantage.

Given a CFPTP forward key $\ForwKey$ and input $y$ (from the permutation domain $\Domain$), $\Adversary_{\Hyb}$ works as follows:
\begin{itemize}
	\item Run the environment that distinguishes $\Real', \Hyb$.
	\item If $\Environment$ observes $\Simulator_\Hyb$ aborting, output $x, x'$, where $x$ is recovered from the $\CommitMsg$ message $(y, c_0, c_1)$, i.e. $x = \PKE.\Dec_{\DecKey}(c_b) = \CFPTP.\Back_{b, \BackKey}(y)$. Meanwhile $x'$ is recovered from the $\OpenMsg$ message $(b', x', r)$, i.e. $x' = \CFPTP.\Back_{b', \BackKey}(y)$, $c_b' = \PKE.\Enc_{\EncKey}(x'; r)$ and $b\ne b'$. Otherwise, output two random elements in $\Domain$.
\end{itemize}
This succeeds with (at least) the probability $\Simulator_\Hyb$ aborts.

\subsection{Easycrypt Proof for showing $\Real' \approx \Hyb$}\label{sec:cfptp_easycrypt}
Since $\Environment$'s view is identical between $\Real'$ and $\Hyb$ except when $\Simulator_\Hyb$ aborts, most of the easycrypt proof should be a series of rewritings and transformations that do not give $\Environment$ any distinguishing advantage. Then, we can upper bound $\Environment$'s distinguishing advantage by the probability of the ``bad event'', i.e. when $\Simulator_\Hyb$ finds a claw for the pair of trapdoor permutations. We can express this via the following lemma:

\begin{easycrypt}[label=code:lem_real_prime_equiv_hyb, caption=Lemma for indistinguishability between $\Real'$ and $\Hyb$ views.]
lemma REAL'_HYB &m :
`|Pr[REAL'.main() @ &m : res] - Pr[HYB.main() @ &m : res]|
<= Pr[CFP_Game(CFAdv).main() @ &m : res].
\end{easycrypt}

\megan{\code{HYB} is not currently written anywhere. It can be expressed either using the DSL or done directly in the generated easycrypt code.}

We also need the following helper lemmas:

\begin{easycrypt}[label=code:lem_real_prime_equiv_hyb_main, caption={Lemma stating that if between $\Real'$ and $\Hyb$ are indistinguishable, then if $\Hyb$ doesn't abort, the results of $\Real'$ and $\Hyb$ are the same. This helps to prove \cref{code:up_to_bad_claim}.}]
lemma REAL'_HYB_main &m :
equiv[REAL'.main ~ HYB.main : true => ! HYB.bad{2} => ={ res } ].
\end{easycrypt}

\megan{maybe add a diagram to show that the programs stay the same, except when bad event happens}

\begin{easycrypt}[label=code:lem_hyb_main_abort_ub, caption={Lemma stating that the probability the bad event occurs in $\Hyb$ is less than or equal to the probability some claw-finding adversary wins the claw-finding game (\cref{code:mod_cfgame}).}]
lemma HYB_main_abort_ub &m :
Pr[HYB.main() @ &m : HYB.bad == true] <= Pr[CFP_Game(CFAdv).main() @ &m : res].
\end{easycrypt}

To prove this, we need to instantiate a claw-finding adversary.

\subsubsection{Claw-free pair game}

First, in ${\sf Cfptp.ec}$, we defined the claw-free pair game is defined as

\begin{easycrypt}[label=code:mod_cfgame, caption={Module representing the claw-finding game.}]
module CFP_Game(Cf: ClawFinder) = {
	proc main(): bool = {
		var fk: fkey; var bk: bkey;
		var x0, x1 : D;

		(fk, bk) <$ keygen;             (* Generates keys for CFPTP. $*)
		(x0, x1) <@ Cf.find_claw(fk);   (* Find any claw for the CFPTP *)
		return (forw fk x0 false = forw fk x1 true); (* Cf succeeds when this happens *)
	}
}.
\end{easycrypt}

\subsubsection{Defining a claw-finding adversary}

Now, define the claw-finding module type:

\begin{easycrypt}[label=code:mod_type_cfadv, caption={Module type for a claw-finding adversary.}]
module type ClawFinder = {
	proc find_claw(fk: fkey) : (D * D)
}.
\end{easycrypt}

Next, define a claw-finding adversary which runs $\Simulator_\Hyb$'s execution and outputs a claw if $\Simulator_\Hyb$ aborts.

\begin{easycrypt}[label=code:mod_cfadv, caption={Module instantiating a claw-finding adversary using access to an environment distinguishing $\Real'$ and $\Hyb$.}]
module CFAdv : ClawFinder ( (* has access to Environment for Real' vs Hyb game *) ) = {
	var Sim.x, Sim.x' : D;
	proc find_claw(fk : fkey) : D * D = {

		(* Run environment for Real' vs Hyb game, with fk in the CRS. *)

		if ( (* Environment sees that Hyb.bad == true *) ) {
			(* Finding a claw means: forw fk x false = forw fk x true = y *)
			return (Sim.x, Sim.x'); (* Get and return the claw from Sim's memory. *)
		}
		else {
			return (Sim.y, Sim.y); (* Failure to find claw. Equivalent to randomly guessing two elements. *)
		}
	}
}.
\end{easycrypt}

\section{Hybrid \ref{hyb:2}: Showing $\Hyb \approx \Ideal$}\label{sec:hyb2}

\subsection{Overview}\label{sec:indcpa_overview}
IND-CPA (i.e. semantic) security in a PKE scheme captures the property that the scheme's ciphertexts reveal only a negligible amount of information about their underlying plaintexts. We show that any adversary that distinguishes between $\Hyb$ and $\Ideal$ can be used to win the IND-CPA game (as described in \cref{sec:indcpa_easycrypt}) via a cryptographic reduction.

The outline for the reduction works as follows: Assume that there exists an $\Environment$ that distinguishes between $\Hyb$ and $\Ideal$ with probability $p$. Next, construct an adversary $\IndcpaAdversary$ that wins the IND-CPA security game with the same probability.

First, we describe the difference between $\Hyb$ and $\Ideal$. In $\Hyb$, observe that the $\CommitMsg$ message $(y, c_0, c_1)$, given the committed bit $b$, is
\begin{itemize}
	\item $y = \CFPTP.\Forw_{b, \ForwKey}(x)$
	\item $c_b = \PKE.\Enc_\EncKey(x; r_b)$
	\item $c_{1-b} = \PKE.\Enc_\EncKey(0; r_{1-b})$.
\end{itemize}
where $x, r_0, r_1$ are uniformly sampled.

In $\Ideal$, the $\CommitMsg$ message $(y, c_0, c_1)$ is:
\begin{itemize}
	\item $y$ sampled uniformly in $\Domain$
	\item $c_0 = \PKE.\Enc_\EncKey(x_0; r_0)$ such that $x_0 = \CFPTP.\Back_{0, \BackKey}(y)$
	\item $c_1 = \PKE.\Enc_\EncKey(x_1; r_1)$ such that $x_1 = \CFPTP.\Back_{1, \BackKey}(y)$
\end{itemize}
where $r_0, r_1$ are uniformly sampled and the committed bit $b$ is unknown.

The difference between $\Hyb$ and $\Ideal$ is the encrypted plaintext for $c_{1-b}$. In essence, distinguishing $\Hyb$ and $\Ideal$, means distinguishing between whether $c_{1-b}$ is an encryption of $0$ or $x_{1-b}$.

Hence, given an environment $\Environment$ that distinguishes between $\Hyb$ and $\Ideal$ with probability $p$, we can construct an adversary $\IndcpaAdversary$ that breaks IND-CPA security with at least the same probability. On input the committed bit $b$, $\IndcpaAdversary$ interacts with to a challenger $\IndcpaGM$ and does the following:
\begin{enumerate}
	\item Sample $y \gets \Domain$ and $r_0, r_1$ uniformly from the randomness distribution.
	\item Compute $x_0 = \CFPTP.\Back_{0, \BackKey}(y)$ and $x_1 = \CFPTP.\Back_{1, \BackKey}(y)$.
	\item Compute $c_b = \PKE.\Enc_{\EncKey}(x_b; r_b)$.
	\item Send $(0, x_{1-b})$ to $\IndcpaGM$, and receive an encryption $c^*$ of either $0$ or $x_{1-b}$.
	\item Set $c_{1-b} = c^*$.
	\item Continue the rest of $\Hyb'$'s execution to generate the rest of $\Environment$'s view. (The view generated by $\Ideal'$ will be exactly the same.)
	\item Send the view generated by $\Hyb'$ to $\Environment$, which outputs a bit indicating ``$\Hyb'$'' or ``$\Ideal'$''.
	\item Output $\Environment$'s bit.
\end{enumerate}

Here, if $\Environment$ sees that $c_{1-b} = \PKE.\Enc_\EncKey(0; r_{1-b})$, it will output $\Hyb$. When $\Environment$ sees that $c_{1-b} = \PKE.\Enc_\EncKey(x_{1-b}; r_{1-b})$, it will output $\Ideal$. Hence, $\IndcpaAdversary$ outputs the correct bit with exactly the same probability that $\IndcpaAdversary$ distinguishes between $\Hyb$ and $\Real$.

\subsection{Mini Hybrids}
To accomplish the reduction to claw-free property of the CFPTP scheme, we require the following hybrid steps:
\begin{enumerate}
	\item\label{hyb:2.1} $\Hyb \approx \Hyb'$ - In $\Hyb'$, switch from sampling $x\gets \Domain$ then running $\CFPTP.\Forw_b$ to get $y$ to sampling $y \gets \Domain$ then running $\CFPTP.\Back$ to get $x_0,x_1$. Next, use $x_b$ to compute the encryption $c_b$ (and discard $x_{1-b}$). Compute $c_{1-b}$ as an encryption of $0$.
	Also, we need to prove that sampling $y$ then generating $x_b = \CFPTP.\Back_b(y)$ is equivalent to sampling $x$ then generating $y = \CFPTP.\Forw_b(x_b)$. In other words, we can conclude ${=\{y, x_b\}}$ in easycrypt. \megan{Relevant tactics are in \code{perm-rnd.ec}.}

	Further, we need to eagerly sample the CRS at the beginning of $\Hyb'$'s execution such that $\ForwKey, \BackKey, \EncKey, \DecKey$ are saved in global variables. Thus, the committer can access $\BackKey$. This shift doesn't affect $\Hyb'$'s output, since the CRS is still sampled the same way using the CFPTP and PKE key generation algorithms. Additionally, we need to eagerly one of the encryption randomness values,specifically $r_{1-b}$ from $\Hyb$, to match how sampling occurs in the IND-CPA game (\cref{line:eager_r0,line:eager_r1} in \cref{code:indcpa_mods}).

	\item\label{hyb:2.2} $\Hyb' \approx \Ideal'$ - Using the IND-CPA security of the PKE scheme, compute $c_{1-b}$ as the encryption of $x_{1-b}$, not $0$. To fit with the IND-CPA adversary, the bit $b$ is still known, as the ciphertext $c_b$ is still $\PKE.\Enc_{\EncKey}(x_b; r_b)$. Discussion about expressing this in easycrypt is in \cref{sec:indcpa_easycrypt}.

	\item\label{hyb:2.3} $\Ideal' \approx \Ideal$ - The final step to $\Ideal$ now works uniformly, i.e. without knowing the value of $b$. In other words, we argue that an $\Ideal$ view indistinguishable from $\Ideal'$ can be generated by a $\Simulator$ who doesn't know the bit $b$. This requires arguing that $\Ideal'$'s ciphertexts $c_b, c_{1-b}$ are uniform in the ciphertext distribution (\cref{sec:ideal_prime_equiv_ideal}).
\end{enumerate}

\subsection{Easycrypt Proof for showing $\Hyb' \approx \Ideal'$}\label{sec:indcpa_easycrypt}
Following the logic from \cref{sec:indcpa_overview}, our goal is to show that the $\Environment$'s distinguishing advantange between $\Hyb'$ and $\Ideal'$ is:

\begin{easycrypt}[label=code:lem_hyb_prime_ideal_prime, caption={Lemma for indistinguishability between the HYB' and IDEAL' views, relying on the IND-CPA security of the PKE scheme.}]
lemma HYB'_IDEAL' &m :
`|Pr[HYB'.main() @ &m : res] - Pr[IDEAL'.main() @ &m : res]|
<= `|Pr[INDCPA_0(INDCPAAdv).main() @ &m : res] - Pr[INDCPA_1(INDCPAAdv).main() @ &m : res]|.
\end{easycrypt}

\subsubsection{Formalizing the IND-CPA Game}
The IND-CPA game is formalized in ${\sf Pke.ec}$ via defining two modules \code{INDCPA\_0}, \code{INDCPA\_1} that respectively encrypt two adversarially chosen plaintexts $x_0, x_1$.

The proof will ultimately show that $\Hyb'$ is equivalent to \code{INDCPA\_0(INDCPAAdv)} and \code{INDCPA\_1(INDCPAAdv)} is equivalent to $\Ideal'$.

\begin{easycrypt}[label=code:indcpa_mods, caption={Modules that capture the IND-CPA adversary's task: guess whether ciphertext 0 or 1 was encrypted.}]
module INDCPA_0(INDCPAAdv : ADV_INDCPA) = { (* Always encrypts x0 *)
	proc main() : bool = {
		var pk : pkey; var sk : skey;
		var r : rand;
		var x0, x1 : plaintext;
		var b : bool;
		var c : ciphertext;
		(pk, sk) <$ dkeygen;         		(* Generate keys for PKE *)
		r <$ drand;                  		(* Select randomness used in PKE *)/*\label{line:eager_r0}*/
		(x0, x1) <@ INDCPAAdv.choose(pk);	(* Adv chooses plaintexts x0, x1 *)
		c <- enc pk x0 r;            		(* Encrypt x0 *)
		b <@ INDCPAAdv.main(pk, c);			(* Adv guesses which ciphertext was encrypted *)
		return b;
	}
}.

module INDCPA_1(INDCPAAdv : ADV_INDCPA) = { (* Always encrypts x1*)
	proc main() : bool = {
		var pk : pkey; var sk : skey;
		var r : rand;
		var x0, x1 : plaintext;
		var b : bool;
		var c : ciphertext;
		(pk, sk) <$ dkeygen;         		(* Generate keys for PKE *)
		r <$ drand;                  		(* Select randomness used in PKE *)/*\label{line:eager_r1}*/
		(x0, x1) <@ INDCPAAdv.choose(pk);	(* Adv chooses plaintexts x0, x1 *)
		c <- enc pk x1 r;            		(* Encrypt x1 *)
		b <@ INDCPAAdv.main(pk, c);			(* Adv guesses which ciphertext was encrypted *)
		return b;
	}
}.
\end{easycrypt}

Then, an IND-CPA adversary wins if it correctly guesses which module (\code{INDCPA\_0}, \code{INDCPA\_1}) it is in. We formalize this idea as the adversary's ``advantage'', which is captured as:

\begin{easycrypt}[label=code:indcpa_adv, caption={Given an initial memory \code{m}, captures the distance between a) the probability \code{INDCPA\_0.main()} running with the \code{INDCPAAdv} returns true and b) the probability \code{INDCPA\_1.main()} running with the \code{INDCPAAdv} returns true. }]
`|Pr[INDCPA_0(INDCPAAdv).main() @ &m : res] - Pr[INDCPA_1(INDCPAAdv).main() @ &m : res]|
\end{easycrypt}

\subsubsection{The game $\G$}
To appropriately model an IND-CPA adversary, we need the ability to suspend and resume a protocol execution. In particular, we need to instantiate a game $\G$ that works as follows:

\parhead{$\G$:}
\begin{enumerate}
	\item $\G$ is parameterized by the UC $\Environment$ (and $\Adversary$) for distinguishing $\Hyb'$ vs $\Ideal'$. $\G$ works as follows:
	\begin{itemize}
		\item If $\Adversary$ corrupts the committer (or if $\Environment$ gets or has control and simply decides to return its Boolean result early), run the protocol execution to completion and output $\Environment$'s final Boolean judgment that guesses whether the execution is $\Hyb'$ or $\Ideal'$.
		\item If the committer is honest, run the protocol execution up until the computation $c_{1-b} = \PKE.\Enc_{\EncKey}(0; r_{1-b})$. Rather than computing $c_{1-b}$, $\G$ \emph{suspends} the protocol's execution and returns $x_{1-b}$ as a global variable.

		Additionally, $\G$ needs the ability to \emph{resume} the protocol's execution after suspension, upon receiving an input that is the encryption of either $0$ or $x_{1-b}$ and setting $c_{1-b}$ to be this input. Then, the protocol runs to completion and produces $\Environment$'s final Boolean judgment that guesses whether the execution is $\Hyb'$ or $\Ideal'$.

	\end{itemize}
\end{enumerate}

\subsubsection{The IND-CPA adversary \code{INDCPAAdv}}
Using $\G$, we build a module for the IND-CPA adversary \code{INDCPAAdv} that works as follows:

\begin{easycrypt}[label=code:indcpaadv, caption={Module representing an IND-CPA adversary constructed using the game $\G$.}]
type g_out = [		(* Type for G.start()'s possible outputs *)
	| bool			(* G.start() output a bool indicating HYB' or IDEAL'*)
	| plaintext		(* G.start() output a x_{1-b} *)
].

module INDCPAAdv : ADV_INDCPA ( (* Environment distinguishing HYB' vs IDEAL' *) ) = {
	module G = G( (* Environment distinguishing HYB' vs IDEAL' *) )

	(* Global vars *)
	var res : bool option; (* Environment's guess, which is returned by G *)

	proc choose(pk : pkey) : plaintext * plaintext = {
		var result : g_out;
		(* Start G's execution. If the committer is corrupted, return the Environment's bool guess of HYB' vs IDEAL'. If the committer is honest, generate the view of the protocol up to computing c_{1-b} and output a plaintext x_{1-b}. *)
		result <@ G.start();

		match result with
		| bool b -> { /*\label{line:unif_ptxts_start}*/
			(* If result is a bool, save b globally. *)
			res <- b;

			(* Then, sample and return two random plaintexts. *)
			p0 <$ dplaintext;
			p1 <$ dplaintext;
			return (p0, p1);
		} /*\label{line:unif_ptxts_end}*/
		| Pke.plaintext x -> {
			(* If result is a plaintext, then note that G suspended. Then, we need to resume the computation in main. *)
			return (zero, x);
		}
	}

	proc main(pk : pkey, c : ciphertext) : bool = { (* c is a ciphertext encrypting either zero or x *)
		match res with
		| Some guess -> {
			(* Environment in G has already guessed HYB' or IDEAL' *)
			return guess;
		}
		| None -> {
			(* Resume the protocol execution until it completes completion. Save the environment's bool guess for HYB' vs IDEAL'. *)
			guess <- G.resume(c);

			if ( guess = 0 ) {
				return 0;	(* Guess that c is an encryption of 0 *)
			}
			else ( guess = 1 ) {
				return 1;	(* Guess that c is an encryption of plaintext x *)
			}
		}
	}
}.
\end{easycrypt}

\subsubsection{Note about $\Ideal' \approx \Ideal$}\label{sec:ideal_prime_equiv_ideal}
To see that $\Ideal' \approx \Ideal$, note that \code{INDCPA\_1(INDCPAAdv)} knows $b$ (assuming we get to that point of the computation), but the way it computes the ciphertexts $c_0, c_1$ is uniform in $b$. The underlying plaintexts $x_b, x_{1-b}$ are sampled uniformly (see \cref{line:unif_ptxts_start} to \cref{line:unif_ptxts_end} in \cref{code:indcpaadv}). Also, the encryption randomness $r_b, r_{1-b}$ are sampled uniformly. Consequently, the encryptions of $x_b, x_{1-b}$ are uniform values in the ciphertext distribution. This allows us to argue that $\Ideal' \approx \Ideal$ when $\Ideal$ doesn't know the committed bit $b$.

{\small{
\bibliographystyle{alpha}
\bibliography{refs}
}}

\end{document}
